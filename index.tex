\documentclass[spanish,]{article}
\usepackage{lmodern}
\usepackage{amssymb,amsmath}
\usepackage{ifxetex,ifluatex}
\usepackage{fixltx2e} % provides \textsubscript
\ifnum 0\ifxetex 1\fi\ifluatex 1\fi=0 % if pdftex
  \usepackage[T1]{fontenc}
  \usepackage[utf8]{inputenc}
\else % if luatex or xelatex
  \ifxetex
    \usepackage{mathspec}
  \else
    \usepackage{fontspec}
  \fi
  \defaultfontfeatures{Ligatures=TeX,Scale=MatchLowercase}
\fi
% use upquote if available, for straight quotes in verbatim environments
\IfFileExists{upquote.sty}{\usepackage{upquote}}{}
% use microtype if available
\IfFileExists{microtype.sty}{%
\usepackage[]{microtype}
\UseMicrotypeSet[protrusion]{basicmath} % disable protrusion for tt fonts
}{}
\PassOptionsToPackage{hyphens}{url} % url is loaded by hyperref
\usepackage[unicode=true]{hyperref}
\hypersetup{
            pdftitle={Historia y filosofía de la ciencia},
            pdfborder={0 0 0},
            breaklinks=true}
\urlstyle{same}  % don't use monospace font for urls
\usepackage[margin=1in]{geometry}
\ifnum 0\ifxetex 1\fi\ifluatex 1\fi=0 % if pdftex
  \usepackage[shorthands=off,main=spanish]{babel}
\else
  \usepackage{polyglossia}
  \setmainlanguage[]{spanish}
\fi
\usepackage{longtable,booktabs}
% Fix footnotes in tables (requires footnote package)
\IfFileExists{footnote.sty}{\usepackage{footnote}\makesavenoteenv{long table}}{}
\usepackage{graphicx,grffile}
\makeatletter
\def\maxwidth{\ifdim\Gin@nat@width>\linewidth\linewidth\else\Gin@nat@width\fi}
\def\maxheight{\ifdim\Gin@nat@height>\textheight\textheight\else\Gin@nat@height\fi}
\makeatother
% Scale images if necessary, so that they will not overflow the page
% margins by default, and it is still possible to overwrite the defaults
% using explicit options in \includegraphics[width, height, ...]{}
\setkeys{Gin}{width=\maxwidth,height=\maxheight,keepaspectratio}
\IfFileExists{parskip.sty}{%
\usepackage{parskip}
}{% else
\setlength{\parindent}{0pt}
\setlength{\parskip}{6pt plus 2pt minus 1pt}
}
\setlength{\emergencystretch}{3em}  % prevent overfull lines
\providecommand{\tightlist}{%
  \setlength{\itemsep}{0pt}\setlength{\parskip}{0pt}}
\setcounter{secnumdepth}{0}
% Redefines (sub)paragraphs to behave more like sections
\ifx\paragraph\undefined\else
\let\oldparagraph\paragraph
\renewcommand{\paragraph}[1]{\oldparagraph{#1}\mbox{}}
\fi
\ifx\subparagraph\undefined\else
\let\oldsubparagraph\subparagraph
\renewcommand{\subparagraph}[1]{\oldsubparagraph{#1}\mbox{}}
\fi

% set default figure placement to htbp
\makeatletter
\def\fps@figure{htbp}
\makeatother

\usepackage{etoolbox}
\makeatletter
\providecommand{\subtitle}[1]{% add subtitle to \maketitle
  \apptocmd{\@title}{\par {\large #1 \par}}{}{}
}
\makeatother
\usepackage{fontspec}
\setmainfont{Adobe Jenson Pro}
\linespread{1.05}
% https://github.com/rstudio/rmarkdown/issues/337
\let\rmarkdownfootnote\footnote%
\def\footnote{\protect\rmarkdownfootnote}

% https://github.com/rstudio/rmarkdown/pull/252
\usepackage{titling}
\setlength{\droptitle}{-2em}

\pretitle{\vspace{\droptitle}\centering\huge}
\posttitle{\par}

\preauthor{\centering\large\emph}
\postauthor{\par}

\predate{\centering\large\emph}
\postdate{\par}
\usepackage{booktabs}
\usepackage{longtable}
\usepackage{array}
\usepackage{multirow}
\usepackage{wrapfig}
\usepackage{float}
\usepackage{colortbl}
\usepackage{pdflscape}
\usepackage{tabu}
\usepackage{threeparttable}
\usepackage{threeparttablex}
\usepackage[normalem]{ulem}
\usepackage{makecell}
\usepackage{xcolor}

\title{Historia y filosofía de la ciencia}
\date{}

\begin{document}
\maketitle

\subsection{Descripción del curso}\label{descripciuxf3n-del-curso}

\begin{longtable}[]{@{}llll@{}}
\toprule
Right Le & ft & Center De & fault\tabularnewline
\midrule
\endhead
12 & 12 & 12 & 12\tabularnewline
123 & 123 & 123 & 123\tabularnewline
1 & 1 & 1 & 1\tabularnewline
\bottomrule
\end{longtable}

La filosofía de la ciencia trata varios problemas filosóficos que surgen
al interior de la ciencia y la práctica científica. Este curso es una
introducción a estas preguntas alrededor de tres cuestiones:

\begin{enumerate}
\def\labelenumi{\arabic{enumi}.}
\tightlist
\item
  ¿Qué es la ciencia?
\item
  ¿Qué es el progreso científico?
\item
  ¿Cuál es el lugar de la ciencia en la sociedad?
\end{enumerate}

\begin{verbatim}
<th>
<img align="right" src="img/1.jpg" title="Unknown + Heikenwaelder Hugo [CC BY-SA 2.5], via Wikimedia Commons" style="width:350;height:350px;padding: 10px;">
\end{verbatim}

El objetivo central de este curso es familiarizar al estudiante con
preguntas fundamentales de la filosofía de la ciencia general y, si el
tiempo lo permite, algunas de las problemáticas que surgen dentro de las
ciencias particulares y de las que se ocupan las filosofías de las
ciencias específicas.

Para ello, primero, revisaremos un poco de la historia de la física para
tener una aproximación a la pregunta ¿qué es la ciencia? La idea es que
para saber qué es la ciencia tenemos que ver primero algunos casos
concretos de la historia de la ciencia. Para eso estudiaremos algo de la
historia de la física y la astronomía: empezaremos con algo de ciencia
aristotélica y geometría euclidieana, luego veremos un poco de
astronomía con el modelo ptolemáico y el copernicano y terminaremos con
algo de física newtoniana, relativista y cuántica.

Después veremos dos posibles formas de responder a la pregunta por la
naturaleza de la ciencia, la postura del empirismo lógico y la de Thomas
Kuhn. De acuerdo con una, la ciencia es una cálculo axiomático con una
interpretación que la vincula con el mundo. De acuerdo con la otra, la
ciencia es una práctica humana específica, con todos los vicios y
virtudes que esto implica.

Finalizaremos con algunas cuestion el rol de la ciencia en la sociedad.
Para ello veremos lo que tiene decir la filosofía de la ciencia
feminista sobre la forma en que debemos entender la influencia del
género en la práctica científica. También veremos una pregunta sobre la
responsabilidad que tiene la ciencia sobre los decubrimientos
científicos y una manera de aproximarse a esta pregunta. Si el tiempo lo
permite, veremos, al final del semestre, algunos problemas en las
ciencias específicas, dependiendo de los intereses particulares de los
estudiantes matriculados.

\textbf{Profesor}: \href{../index.html}{Juan Camilo Espejo-Serna}~

\textbf{Horario}: Martes, 8:00 - 10:00; Viernes 8:00 - 9:00 am.

\textbf{Grupo de MS Teams}: Por definir

\textbf{Cuaderno de OneNote}: Por definir

\begin{center}\rule{0.5\linewidth}{\linethickness}\end{center}

\subsection{Objetivos}\label{objetivos}

\begin{itemize}
\item
  Dominar el lenguaje propio de la discusión filosófica sobre la ciencia
  para aportar en las discusiones sobre la naturaleza de la ciencia, del
  progreso científico y su lugar en la sociedad.
\item
  Distinguir, relacionar y sistematizar conocimientos aportados por la
  ciencia específicas y la filosofía para dar cuenta de la naturaleza de
  la ciencia, del progreso científico y su lugar en la sociedad.
\item
  Plantear autónoma y críticamente relaciones entre distintos fenómenos
  científicos para interpretarlos y establecer vínculos entre el
  conocimiento teórico y la práctica científica.
\item
  Utilizar TIC para apoyar el estudio filosófico de la ciencia.
\end{itemize}

\begin{center}\rule{0.5\linewidth}{\linethickness}\end{center}

\subsection{Metodología}\label{metodologuxeda}

\paragraph{\texorpdfstring{\textbf{Antes de la
sesión}}{Antes de la sesión}}\label{antes-de-la-sesiuxf3n}

\begin{itemize}
\tightlist
\item
  Todos los estudiantes deberán subir un control de lectura por tarde
  \textbf{75 horas} antes de la sesión.
\end{itemize}

\paragraph{\texorpdfstring{\textbf{Durante la
sesión}}{Durante la sesión}}\label{durante-la-sesiuxf3n}

\begin{itemize}
\item
  Todos deben atender con cuidado a la presentación del profesor y
  formular preguntas al respecto. Los controles de lectura transforman
  la clase en la medida en que las presentaciones se irán ajustando a lo
  que ustedes reflejen en los controles de lectura.
\item
  Revisen si entienden la exposición y si están de acuerdo; pregunten
  por las relaciones con los temas anteriormente expuestos.
\end{itemize}

\begin{center}\rule{0.5\linewidth}{\linethickness}\end{center}

\subsection{Plan semanal}\label{plan-semanal}

\subsubsection{Semana 1}\label{semana-1}

\begin{tabular}{>{\raggedright\arraybackslash}p{30em}|>{\raggedright\arraybackslash}p{30em}}
\hline
Martes & Viernes\\
\hline
Presentaci<U+00F3>n del programa & Las preguntas de la filosof<U+00ED>a de la ciencia\\
\hline
\end{tabular}

Presentación en pantalla completa

\begin{center}\rule{0.5\linewidth}{\linethickness}\end{center}

\subsubsection{Semana 2}\label{semana-2}

\begin{tabular}{>{\raggedright\arraybackslash}p{30em}|>{\raggedright\arraybackslash}p{30em}}
\hline
Martes & Viernes\\
\hline
Expliaci<U+00F3>n aristot<U+00E9>lica & Los inicios del m<U+00E9>todo axiom<U+00E1>tico y la explicaci<U+00F3>n aristot<U+00E9>lica\\
\hline
\end{tabular}

\begin{itemize}
\item
  Leer: Losee, J. (1976). Introducción histórica a la filosofía de la
  ciencia. España: Alianza Editorial. Pags. 15-38
\item
  Hacer: Control de lectura
\end{itemize}

Presentación en pantalla completa

\begin{center}\rule{0.5\linewidth}{\linethickness}\end{center}

\subsubsection{Semana 3}\label{semana-3}

\begin{tabular}{>{\raggedright\arraybackslash}p{30em}|>{\raggedright\arraybackslash}p{30em}}
\hline
Martes & Viernes\\
\hline
El universo seg<U+00FA>n el aristotelismo & Cr<U+00ED>ticas al aristotelismo\\
\hline
\end{tabular}

\begin{itemize}
\item
  Leer: Losee, J. (1976). Introducción histórica a la filosofía de la
  ciencia. España: Alianza Editorial. Pags. 53-103
\item
  Hacer: Control de lectura
\end{itemize}

Presentación en pantalla completa

\begin{center}\rule{0.5\linewidth}{\linethickness}\end{center}

\subsubsection{Semana 4}\label{semana-4}

\begin{tabular}{>{\raggedright\arraybackslash}p{30em}|>{\raggedright\arraybackslash}p{30em}}
\hline
Martes & Viernes\\
\hline
F<U+00ED>sica cl<U+00E1>sica & Open Campus\\
\hline
\end{tabular}

\begin{itemize}
\item
  Leer: (*) Fine, A. (1986). The Shaky Game. Chicago, USA: The
  University of Chicago Press. Caps 1 y 3
\item
  Hacer: Control de lectura
\end{itemize}

Presentación en pantalla completa

\begin{center}\rule{0.5\linewidth}{\linethickness}\end{center}

\subsubsection{Semana 5}\label{semana-5}

\begin{tabular}{>{\raggedright\arraybackslash}p{30em}|>{\raggedright\arraybackslash}p{30em}}
\hline
Martes & Viernes\\
\hline
F<U+00ED>sicas no-clasicas: relativista & F<U+00ED>sicas no-clasicas: cu<U+00E1>ntica\\
\hline
\end{tabular}

\begin{itemize}
\item
  Leer: (*) Fine, A. (1986). The Shaky Game. Chicago, USA: The
  University of Chicago Press. Cap 5
\item
  Hacer:
\end{itemize}

Presentación en pantalla completa

\begin{center}\rule{0.5\linewidth}{\linethickness}\end{center}

\subsubsection{Semana 6}\label{semana-6}

\begin{tabular}{>{\raggedright\arraybackslash}p{30em}|>{\raggedright\arraybackslash}p{30em}}
\hline
Martes & Viernes\\
\hline
Repaso & Empirismo l<U+00F3>gico\\
\hline
\end{tabular}

\begin{itemize}
\item
  Leer: Suppe, Frederick (1979)~La estructura de las teorías
  científicas. Editora Nacional: Madrid, España. Partes I, II (★) y III
\item
  Hacer: Control de lectura y Taller virtual
\end{itemize}

Presentación en pantalla completa

\begin{center}\rule{0.5\linewidth}{\linethickness}\end{center}

\subsubsection{Semana 7}\label{semana-7}

\begin{tabular}{>{\raggedright\arraybackslash}p{30em}|>{\raggedright\arraybackslash}p{30em}}
\hline
Martes & Viernes\\
\hline
Empirismo l<U+00F3>gico & Empirismo l<U+00F3>gico\\
\hline
\end{tabular}

\begin{itemize}
\item
  Leer: Suppe, Frederick (1979)~La estructura de las teorías
  científicas. Editora Nacional: Madrid, España. Partes I, II (★) y III
\item
  Hacer: Control de lectura
\end{itemize}

Presentación en pantalla completa

\begin{center}\rule{0.5\linewidth}{\linethickness}\end{center}

\subsubsection{Semana 8}\label{semana-8}

\begin{tabular}{>{\raggedright\arraybackslash}p{30em}|>{\raggedright\arraybackslash}p{30em}}
\hline
Martes & Viernes\\
\hline
Empirismo l<U+00F3>gico & Empirismo l<U+00F3>gico\\
\hline
\end{tabular}

\begin{itemize}
\item
  Leer: Suppe, Frederick (1979)~La estructura de las teorías
  científicas. Editora Nacional: Madrid, España. Partes I, II y III (★)
\item
  Hacer: Control de lectura
\end{itemize}

Presentación en pantalla completa

\begin{center}\rule{0.5\linewidth}{\linethickness}\end{center}

\subsubsection{Semana 9}\label{semana-9}

\begin{tabular}{>{\raggedright\arraybackslash}p{30em}|>{\raggedright\arraybackslash}p{30em}}
\hline
Martes & Viernes\\
\hline
Problemas del empirismo l<U+00F3>gico & Quine: Dos dogmas del empirismo\\
\hline
\end{tabular}

\begin{itemize}
\item
  Leer: Quine, W. V. O. (2002) Desde un punto de vista lógico. Paidos:
  Barcelona, España. Cap. 2
\item
  Hacer: Control de lectura
\end{itemize}

Presentación en pantalla completa

\begin{center}\rule{0.5\linewidth}{\linethickness}\end{center}

\subsubsection{Semana 10}\label{semana-10}

\begin{tabular}{>{\raggedright\arraybackslash}p{30em}|>{\raggedright\arraybackslash}p{30em}}
\hline
Martes & Viernes\\
\hline
Popper: contra el empirismo l<U+00F3>gico & Popper: conjeturas y refutaciones\\
\hline
\end{tabular}

\begin{itemize}
\item
  Leer: Popper, K (2002)~Conjeturas y refutaciones. Paidos: Barcelona,
  España. Cap 1 (★)
\item
  Hacer: Control de lectura
\end{itemize}

Presentación en pantalla completa

\begin{center}\rule{0.5\linewidth}{\linethickness}\end{center}

\subsubsection{Semana 11}\label{semana-11}

\begin{tabular}{>{\raggedright\arraybackslash}p{30em}|>{\raggedright\arraybackslash}p{30em}}
\hline
Martes & Viernes\\
\hline
Repaso & Kuhn: ciencia normal\\
\hline
\end{tabular}

\begin{itemize}
\item
  Leer: Kuhn, Thomas (1962)~La estructura de las revoluciones
  científicas. Fondo de cultura económica: México. Caps 1, 2, 3 y 4
\item
  Hacer: Control de lectura y el taller en virtual sabana
\end{itemize}

Presentación en pantalla completa

\begin{center}\rule{0.5\linewidth}{\linethickness}\end{center}

\subsubsection{Semana 12}\label{semana-12}

\begin{tabular}{>{\raggedright\arraybackslash}p{30em}|>{\raggedright\arraybackslash}p{30em}}
\hline
Martes & Viernes\\
\hline
Kuhn: ciencia normal & Kuhn: ciencia normal\\
\hline
\end{tabular}

\begin{itemize}
\item
  Leer: Kuhn, Thomas (1962)~La estructura de las revoluciones
  científicas. Fondo de cultura económica: México. Caps 5, 6, 7 y 8
\item
  Hacer: Control de lectura
\end{itemize}

Presentación en pantalla completa

\begin{center}\rule{0.5\linewidth}{\linethickness}\end{center}

\subsubsection{Semana 13}\label{semana-13}

\begin{tabular}{>{\raggedright\arraybackslash}p{30em}|>{\raggedright\arraybackslash}p{30em}}
\hline
Martes & Viernes\\
\hline
Kuhn: ciencia revolucionaria & Kuhn: ciencia revolucionaria\\
\hline
\end{tabular}

Presentación en pantalla completa

\begin{center}\rule{0.5\linewidth}{\linethickness}\end{center}

\subsubsection{Semana 14}\label{semana-14}

\begin{tabular}{>{\raggedright\arraybackslash}p{30em}|>{\raggedright\arraybackslash}p{30em}}
\hline
Martes & Viernes\\
\hline
Kuhn: ciencia revolucionaria & Ciencia y valores: Filosof<U+00ED>a feminista de la ciencia\\
\hline
\end{tabular}

\begin{itemize}
\item
  Leer: Harding, Ciencia y Feminismo (Cap 1),
\item
  Hacer: Control de lectura
\end{itemize}

Presentación en pantalla completa

\begin{center}\rule{0.5\linewidth}{\linethickness}\end{center}

\subsubsection{Semana 15}\label{semana-15}

\begin{tabular}{>{\raggedright\arraybackslash}p{30em}|>{\raggedright\arraybackslash}p{30em}}
\hline
Martes & Viernes\\
\hline
Ciencia y valores: Filosof<U+00ED>a feminista de la ciencia & Ciencia y valores: Filosof<U+00ED>a feminista de la ciencia\\
\hline
\end{tabular}

\begin{itemize}
\item
  Leer: Arrieta de Guzmán, Teresa (2018) Sobre el pensamiento feminista
  y la ciencia . En
\item
  Hacer: Control de lectura
\end{itemize}

Presentación en pantalla completa

\begin{center}\rule{0.5\linewidth}{\linethickness}\end{center}

\subsubsection{Semana 16}\label{semana-16}

\begin{tabular}{>{\raggedright\arraybackslash}p{30em}|>{\raggedright\arraybackslash}p{30em}}
\hline
Martes & Viernes\\
\hline
Ciencia y valores: la responsabilidad cient<U+00ED>fica & Ciencia y valores: la responsabilidad cient<U+00ED>fica\\
\hline
\end{tabular}

\begin{itemize}
\item
  Leer: Douglas, Heather E. (2003). The Moral Responsibilities of
  Scientists (Tensions between Autonomy and Responsibility).~American
  Philosophical Quarterly~40 (1):59 - 68.
\item
  Hacer: Control de lectura
\end{itemize}

Presentación en pantalla completa

\begin{center}\rule{0.5\linewidth}{\linethickness}\end{center}

\subsubsection{Semana 17}\label{semana-17}

\begin{tabular}{>{\raggedright\arraybackslash}p{30em}|>{\raggedright\arraybackslash}p{30em}}
\hline
Martes & Viernes\\
\hline
Repaso & Taller final\\
\hline
\end{tabular}

\begin{itemize}
\item
  Leer: ¡Todo!
\item
  Hacer: Repasar todas las presentaciones y Taller final
\end{itemize}

\begin{center}\rule{0.5\linewidth}{\linethickness}\end{center}

\subsection{Evaluación}\label{evaluaciuxf3n}

\paragraph{\texorpdfstring{\textbf{Talleres}}{Talleres}}\label{talleres}

Los talleres consistirán en una serie de preguntas que los alumnos
deberán solucionar en la plataforma virtual. Es deber del estudiante
entender bien cómo funciona la plataforma con anticipación a la fecha
límite de entrega del taller.

\paragraph{\texorpdfstring{\textbf{Control de
lectura}}{Control de lectura}}\label{control-de-lectura}

Extensión: entre 400 y 1000 palabras.

Para cada lectura asignada, los estudiantes deben escribir un texto
corto con la tesis principal, tres afirmaciones/presuposiciones del
texto y tres preguntas/desafíos al texto.

Los controles deberán ser subidos a la plataforma virtual a más tardar
\textbf{75 horas} antes de la sesión. Todos los estudiantes empiezan con
5.0 en esta nota. Por cada vez que no se participe dentro del rango de
tiempo especificado, la nota será disminuida de acuerdo con los
siguientes parámetros: primera vez: -0.5; segunda vez: -1.0; tercera
vez: -1.5; cuarta vez: -2.0.

\paragraph{\texorpdfstring{\textbf{Incumplimiento}}{Incumplimiento}}\label{incumplimiento}

\emph{La vida nos da sorpresas; sorpresas nos da la vida.} Por eso,
todos tienen un control de lectura ``de gracia''. Es decir, pueden dejar
de entregar uno sin problema; el primer control de lectura que les falte
no cuenta. Por ejemplo, si no entregan un control de lectura y entregan
todos los demás, su nota igual queda en 5.0.

Para todo lo demás, es importante avisar al profesor. Hablemos. No me
tienen que contar sus problemas personales pero es importante que si se
encuentran en una situación en la que ven que no pueden cumplir con los
requerimientos de clase me avisen con la mayor anticipación posible y
encontremos un plan para solventar el problema en lo que respecta a la
clase. Insisto: hablemos, no se pierdan \textbf{:)}.

\paragraph{\texorpdfstring{\textbf{Calificación}}{Calificación}}\label{calificaciuxf3n}

\begin{tabular}{l|l|l|l|l|l}
\hline
Taller primer corte & Controles de lectura primer corte & Taller segundo corte & Controles de lectura sugundo corte & Taller tercer corte & Controles de lectura tercer corte\\
\hline
15\% & 15\% & 15\% & 15\% & 25\% & 15\%\\
\hline
\end{tabular}

\end{document}
